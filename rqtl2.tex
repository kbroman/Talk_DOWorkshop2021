\documentclass[12pt,t,aspectratio=169]{beamer}
\usepackage{graphicx}
\setbeameroption{hide notes}
\setbeamertemplate{note page}[plain]
\usepackage{listings}

\input{header.tex}

%%%%%%%%%%%%%%%%%%%%%%%%%%%%%%%%%%%%%%%%%%%%%%%%%%%%%%%%%%%%%%%%%%%%%%
% end of header
%%%%%%%%%%%%%%%%%%%%%%%%%%%%%%%%%%%%%%%%%%%%%%%%%%%%%%%%%%%%%%%%%%%%%%

% title info
\title{R/qtl2}
\subtitle{high-dimensional data and multi-parent populations}
\author{\href{https://kbroman.org}{Karl Broman}}
\institute{Biostatistics \& Medical Informatics, UW{\textendash}Madison}
\date{\href{https://kbroman.org}{\tt \scriptsize \color{foreground} kbroman.org}
\\[-4pt]
\href{https://github.com/kbroman}{\tt \scriptsize \color{foreground} github.com/kbroman}
\\[-4pt]
\href{https://twitter.com/kwbroman}{\tt \scriptsize \color{foreground} @kwbroman}
\\[2pt]
\scriptsize {\lolit Slides:} \href{https://kbroman.org/Talk_DOWorkshop2021}{\tt \scriptsize
  \color{foreground} kbroman.org/Talk\_DOWorkshop2021}
}


\begin{document}

% title slide
{
\setbeamertemplate{footline}{} % no page number here
\frame{
  \titlepage

  \vfill \hfill \includegraphics[height=6mm]{Figs/cc-zero.png}

  \note{These are slides for a talk that I will give for a workshop on
    Diversity Outbred mice on 27 Oct 2021, hosted by the David James
    Lab at the University of Sydney.

    Source: {\tt https://github.com/kbroman/Talk\_DOWorkshop2021} \\
    Slides: {\tt https://kbroman/Talk\_DOWorkshop2021}
}
} }



\begin{frame}[c]{21 years of R/qtl}

\figw{Figs/rqtl_lines_code.pdf}{1.0}

\end{frame}





\begin{frame}[c]{Intercross}
\figw{Figs/intercross.pdf}{1.0}
\end{frame}



\begin{frame}[c]{}

\figh{Figs/rqtl_fig1.png}{0.96}


\hfill
\href{https://doi.org/10.1093/bioinformatics/btg112}{\scriptsize
  \lolit Broman et al. (2003) {\tt doi.org/bsjrwj}}

\end{frame}






\begin{frame}{Good things}

\bbi
  \item efficient handling of missing genotypes
  \item diagnostics and data visualization
  \item fit and exploration of multiple-QTL models
  \item quite comprehensive
  \item quite flexible
\ei

\end{frame}




\begin{frame}{Why work on software?}

  \bbi
\item to be useful
\item makes our own analyses easier
\item platform for implementing new methods
\item has led to many collaborations
  \ei

\end{frame}




\begin{frame}{Bad things}

  \bbi
\item some really bloated code
\item hard to maintain
\item many inconsistencies in the user interface
\item largely restricted to two-parent crosses
  \ei

\end{frame}







\begin{frame}[c]{Improving precision}

  \vspace{-20mm}

  \bbi
\item more recombinations
\item more individuals
\item more precise phenotype
\item lower-level phenotypes
\bi
\item transcripts, proteins, metabolites
  \ei
  \ei

\end{frame}



\begin{frame}[c]{HS/DO}

  \vspace{2mm}

  \figh{Figs/hs.pdf}{0.9}

\end{frame}





\begin{frame}[c]{}

  \vspace*{5mm}

\figh{Figs/rqtl2_3d.png}{0.85}


\vspace{3mm}

\hfill \href{https://kbroman.org/miner_book}{\scriptsize \lolit \tt kbroman.org/miner\_book}

\end{frame}



\begin{frame}[c]{R/qtl2}

\vspace*{-16.2mm}

  \vspace{21mm}

  \bbi
\item High-density genotypes
\item High-dimensional phenotypes
\item Multi-parent populations
\item Linear mixed models
  \ei

  \vspace{25mm}

\hfill \href{https://kbroman.org/qtl2}{\small \tt kbroman.org/qtl2}

\end{frame}



\begin{frame}{QTL mapping in MPPs}

  \bbi
\item Data diagnostics/cleaning
\item Genome reconstruction
\item QTL analysis
\item Visualization/exploration of results
  \ei

\end{frame}


\begin{frame}[c]{Genome reconstruction}

\figh{Figs/genome_reconstr.pdf}{0.95}

\end{frame}




\begin{frame}[c]{QTL mapping in MPPs}

  \bigskip

\figh{Figs/rqtl2_fig1.png}{0.83}


\hfill
\href{https://doi.org/10.1534/genetics.118.301595}{\scriptsize
  \lolit Broman et al. (2019) {\tt doi.org/gfvknr}}

\end{frame}



\begin{frame}[c]{R/qtl2: \color{foreground} Let's not make the same mistakes}

  \bbi
\only<1>{
\item C++ and Rcpp
\item Roxygen2 for documentation
\item Unit tests
\item A single ``switch'' for cross type
}
\only<2>{
{\lolit
\item C++ and Rcpp
\item Roxygen2 for documentation
\item Unit tests
\item A single ``switch'' for cross type
}
}
\onslide<2>{
\item Yet another data input format
\item Flatter data structures, but still complex
}
\ei

\end{frame}




\begin{frame}[c]{}

\Large

{Slides:} \href{https://kbroman.org/Talk_DOWorkshop2021}{\tt
  \color{foreground} kbroman.org/Talk\_DOWorkshop2021} \quad
\includegraphics[height=5mm]{Figs/cc-zero.png}

\vspace{7mm}

\href{https://kbroman.org}{\tt kbroman.org}

\vspace{7mm}

\href{https://kbroman.org/qtl2}{\tt kbroman.org/qtl2}

\vspace{7mm}

\href{https://github.com/kbroman}{\tt github.com/kbroman}

\vspace{7mm}

\href{https://twitter.com/kwbroman}{\tt @kwbroman}


\end{frame}




\end{document}
