\documentclass[12pt,t,aspectratio=169]{beamer}
\usepackage{graphicx}
\setbeameroption{hide notes}
\setbeamertemplate{note page}[plain]
\usepackage{listings}

% header.tex: boring LaTeX/Beamer details + macros

% get rid of junk
\usetheme{default}
\beamertemplatenavigationsymbolsempty
\hypersetup{pdfpagemode=UseNone} % don't show bookmarks on initial view


% font
\usepackage{fontspec}
\setsansfont
  [ ExternalLocation = fonts/ ,
    UprightFont = *-regular ,
    BoldFont = *-bold ,
    ItalicFont = *-italic ,
    BoldItalicFont = *-bolditalic ]{texgyreheros}
\setbeamerfont{note page}{family*=pplx,size=\footnotesize} % Palatino for notes
% "TeX Gyre Heros can be used as a replacement for Helvetica"
% I've placed them in fonts/; alternatively you can install them
% permanently on your system as follows:
%     Download http://www.gust.org.pl/projects/e-foundry/tex-gyre/heros/qhv2.004otf.zip
%     In Unix, unzip it into ~/.fonts
%     In Mac, unzip it, double-click the .otf files, and install using "FontBook"

% named colors
\definecolor{offwhite}{RGB}{255,250,240}
\definecolor{gray}{RGB}{155,155,155}

\ifx\notescolors\undefined % slides
  \definecolor{background}{RGB}{255,255,255}
  \definecolor{foreground}{RGB}{24,24,24}
  \definecolor{title}{RGB}{27,94,134}
  \definecolor{subtitle}{RGB}{22,175,124}
  \definecolor{hilit}{RGB}{122,0,128}
  \definecolor{vhilit}{RGB}{255,0,128}
  \definecolor{lolit}{RGB}{95,95,95}

  \definecolor{myyellow}{rgb}{1,1,0.7}
\else % notes
  \definecolor{background}{RGB}{255,255,255}
  \definecolor{foreground}{RGB}{24,24,24}
  \definecolor{title}{RGB}{27,94,134}
  \definecolor{subtitle}{RGB}{22,175,124}
  \definecolor{hilit}{RGB}{122,0,128}
  \definecolor{vhilit}{RGB}{255,0,128}
  \definecolor{lolit}{RGB}{95,95,95}
\fi
\definecolor{nhilit}{RGB}{128,0,128}  % hilit color in notes
\definecolor{nvhilit}{RGB}{255,0,128} % vhilit for notes

\newcommand{\hilit}{\color{hilit}}
\newcommand{\vhilit}{\color{vhilit}}
\newcommand{\nhilit}{\color{nhilit}}
\newcommand{\nvhilit}{\color{nvhilit}}
\newcommand{\lolit}{\color{lolit}}

% use those colors
\setbeamercolor{titlelike}{fg=title}
\setbeamercolor{subtitle}{fg=subtitle}
\setbeamercolor{institute}{fg=lolit}
\setbeamercolor{normal text}{fg=foreground,bg=background}
\setbeamercolor{item}{fg=foreground} % color of bullets
\setbeamercolor{subitem}{fg=lolit}
\setbeamercolor{itemize/enumerate subbody}{fg=lolit}
\setbeamertemplate{itemize subitem}{{\textendash}}
\setbeamerfont{itemize/enumerate subbody}{size=\footnotesize}
\setbeamerfont{itemize/enumerate subitem}{size=\footnotesize}

% page number
\setbeamertemplate{footline}{%
    \raisebox{5pt}{\makebox[\paperwidth]{\hfill\makebox[20pt]{\lolit
          \scriptsize\insertframenumber}}}\hspace*{5pt}}

% add a bit of space at the top of the notes page
\addtobeamertemplate{note page}{\setlength{\parskip}{12pt}}

% default link color
\hypersetup{colorlinks, urlcolor={hilit}}

\ifx\notescolors\undefined % slides
  % set up listing environment
  \lstset{language=bash,
          basicstyle=\ttfamily\scriptsize,
          frame=single,
          commentstyle=,
          backgroundcolor=\color{darkgray},
          showspaces=false,
          showstringspaces=false
          }
\else % notes
  \lstset{language=bash,
          basicstyle=\ttfamily\scriptsize,
          frame=single,
          commentstyle=,
          backgroundcolor=\color{offwhite},
          showspaces=false,
          showstringspaces=false
          }
\fi

% a few macros
\newcommand{\bi}{\begin{itemize}}
\newcommand{\bbi}{\vspace{24pt} \begin{itemize} \itemsep8pt}
\newcommand{\ei}{\end{itemize}}
\newcommand{\ig}{\includegraphics}
\newcommand{\subt}[1]{{\footnotesize \color{subtitle} {#1}}}
\newcommand{\ttsm}{\tt \small}
\newcommand{\ttfn}{\tt \footnotesize}
\newcommand{\figh}[2]{\centerline{\includegraphics[height=#2\textheight]{#1}}}
\newcommand{\figw}[2]{\centerline{\includegraphics[width=#2\textwidth]{#1}}}


%%%%%%%%%%%%%%%%%%%%%%%%%%%%%%%%%%%%%%%%%%%%%%%%%%%%%%%%%%%%%%%%%%%%%%
% end of header
%%%%%%%%%%%%%%%%%%%%%%%%%%%%%%%%%%%%%%%%%%%%%%%%%%%%%%%%%%%%%%%%%%%%%%

% title info
\title{R/qtl2}
\subtitle{high-dimensional data and multi-parent populations}
\author{\href{https://kbroman.org}{Karl Broman}}
\institute{Biostatistics \& Medical Informatics, UW{\textendash}Madison}
\date{\href{https://kbroman.org}{\tt \scriptsize \color{foreground} kbroman.org}
\\[-4pt]
\href{https://github.com/kbroman}{\tt \scriptsize \color{foreground} github.com/kbroman}
\\[-4pt]
\href{https://twitter.com/kwbroman}{\tt \scriptsize \color{foreground} @kwbroman}
\\[2pt]
\scriptsize {\lolit Slides:} \href{https://kbroman.org/Talk_DOWorkshop2021}{\tt \scriptsize
  \color{foreground} kbroman.org/Talk\_DOWorkshop2021}
}


\begin{document}

% title slide
{
\setbeamertemplate{footline}{} % no page number here
\frame{
  \titlepage

  \vfill \hfill \includegraphics[height=6mm]{Figs/cc-zero.png}

  \note{These are slides for a talk that I will give for a workshop on
    Diversity Outbred mice on 27 Oct 2021, hosted by the David James
    Lab at the University of Sydney.

    Source: {\tt https://github.com/kbroman/Talk\_DOWorkshop2021} \\
    Slides: {\tt https://kbroman/Talk\_DOWorkshop2021}
}
} }



\begin{frame}[c]{21 years of R/qtl}

\figw{Figs/rqtl_lines_code.pdf}{1.0}

\end{frame}





\begin{frame}[c]{Intercross}
\figw{Figs/intercross.pdf}{1.0}
\end{frame}



\begin{frame}[c]{QTL mapping}

\vspace{5mm}
\figw{Figs/lodcurve_insulin_with_effects.pdf}{1.0}
\end{frame}



\begin{frame}{Why?}


\end{frame}



\begin{frame}[c]{}

\centerline{\Large Good things}

\vspace{4mm}

\onslide<2>{
\begin{itemize}
\lolit
  \item some of the code
  \item basics of the user interface
  \item diagnostics and data visualization
  \item quite comprehensive
  \item quite flexible
\end{itemize}
}

\end{frame}


\begin{frame}{}

\vspace*{16.7mm}

\centerline{\Large Bad things}

\end{frame}




\begin{frame}{Challenges}

  \bbi
\item Documentation
\item Support
  \ei

\end{frame}



\begin{frame}[c]{Improving precision}

  \vspace{-20mm}

  \bbi
\item more recombinations
\item more individuals
\item more precise phenotype
\item lower-level phenotypes
\bi
\item transcripts, proteins, metabolites
  \ei
  \ei

\end{frame}



\begin{frame}[c]{HS/DO}

  \vspace{2mm}

  \figh{Figs/hs.pdf}{0.9}

\end{frame}





\begin{frame}[c]{}

  \vspace*{5mm}

\figh{Figs/rqtl2_3d.png}{0.85}


\vspace{3mm}

\hfill \href{https://kbroman.org/miner_book}{\scriptsize \lolit \tt kbroman.org/miner\_book}

\end{frame}



\begin{frame}[c]{R/qtl2}

\vspace*{-16.2mm}

  \vspace{21mm}

  \bbi
\item High-density genotypes
\item High-dimensional phenotypes
\item Multi-parent populations
\item Linear mixed models
  \ei

  \vspace{25mm}

\hfill \href{https://kbroman.org/qtl2}{\small \tt kbroman.org/qtl2}

\end{frame}



\begin{frame}[c]{R/qtl2: \color{foreground} Let's not make the same mistakes}

  \bbi
\only<1>{
\item C++ and Rcpp
\item Roxygen2 for documentation
\item Unit tests
\item A single ``switch'' for cross type
}
\only<2>{
{\lolit
\item C++ and Rcpp
\item Roxygen2 for documentation
\item Unit tests
\item A single ``switch'' for cross type
}
}
\onslide<2>{
\item Yet another data input format
\item Flatter data structures, but still complex
}
\ei

\end{frame}




\begin{frame}[c]{}

\Large

{Slides:} \href{https://kbroman.org/Talk_DOWorkshop2021}{\tt
  \color{foreground} kbroman.org/Talk\_DOWorkshop2021} \quad
\includegraphics[height=5mm]{Figs/cc-zero.png}

\vspace{7mm}

\href{https://kbroman.org}{\tt kbroman.org}

\vspace{7mm}

\href{https://kbroman.org/qtl2}{\tt kbroman.org/qtl2}

\vspace{7mm}

\href{https://github.com/kbroman}{\tt github.com/kbroman}

\vspace{7mm}

\href{https://twitter.com/kwbroman}{\tt @kwbroman}


\end{frame}




\end{document}
